\chapter*{Introduction}\label{cap.introduccion}
\markboth{Introduccion}{Introduccion}
El presente documento detalla la soluci\'on propuesta para la gesti\'on de diversos servicios, actividades y la informaci\'on de personal y clientes que acceden a las instalaciones ofrecidas por las Sucursales del Gimnasio Atl\'etico \textbf{San Pancho}.\\

Se propone un control de acceso mediante torniquete usando la pantalla que provee este dispositivo. Se le proporcionar\'a al Cliente y al Personal una clave de seguridad (NIP) que constar\'a de 4 d\'igitos y un usuario, los cuales podr\'a utilizar para facilitar su acceso a las \'areas permitidas seg\'un su membres\'ia dentro de la sucursal.\\

El documento consta de 2 seciones elementales. \\

Se presenta el an\'alisis del problema a resolver por el sistema, definiendo los requisitos de usuario los cuales nos permitir\'an conocer las necesidades de todas las personas que estar\'an en contacto con el sistema para posteriormente identificar los requisitos de sistema con los cuales se definir\'a la funcionalidad del mismo.\\

Posteriormente, se definir\'a el alcance del problema. Esta secci\'on presenta gran relevancia, ya que permitir\'a conocer los elementos del sistema que presentan una urgencia alta de desarrollo y a su vez, los que se pueden desarrollar en un futuro .

\section{Prop\'osito}
Se desarrollar\'a un sistema con el cual se podr\'a gestionar, dar de alta, dar de baja, actualizar y verificar actividades en las que est\'an inscritas las personas con diferentes membres\'ias. \\
Tambi\'en se pretende poder gestionar los datos de los empleados y el lugar en el cual han sido asignados. 

\section{Definiciones, Acr\'onimos y Abreviaciones}

\textbf{Personal:} Instructores, recepcionista, agente de membres\'ias, que est\'en registrados en el sistema.\\

\textbf{Instructores:} Encargados de una actividad dentro de determinada \'area en el Gimnasio y que est\'e definida en el sistema.\\

\textbf{Recepcionista: }Encargada de recibir a los Clientes, Proveedores, Instructores, dar de alta, modificar o dar baja los clientes, a su vez, es la encargada de resolver dudas a clientes dentro del Gimnasio.\\

\textbf{Agente de Membresías:} Encargado de mostrar las instalaciones a Clientes potenciales, darlos de alta, baja o modificar su membres\'ia.\\

\textbf{Cliente:} Los clientes son todas aquellas personas mayores de 18 años que se encuentran registradas en el sistema del gimnasio y que pueden disfrutar de alguna de sus actividades y servicios.\\

\textbf{Beneficiario:} Los beneficiarios son aquellas personas menores de 18 años que tienen relación con un cliente y además gozan de algunas de las activdades o servicios del gimnasio.\\

\textbf{ID Cliente(PIN personalizado):} Número  de 6 d\'igitos que identifica de manera \'unica al cliente. Servir\'a para saber el estado del cliente.\\

\textbf{Servicio: }Utilidad o funci\'on que brinda el gimnasio para satisfacer las necesidades de los clientes. Por ejemplo: servicio de Lockers, planes de nutrici\'on, regaderas entre otras m\'as.\\

\textbf{Administrador de sistema: }Persona encargada de gestionar las funciones del sistema (dar de alta a un cliente, dar de baja a un cliente, dar de alta un servicio o a un empleado, registro de una membres\'ia etc)\\

\textbf{Actividad en Sucursal: }Din\'amica dentro de un \'area espec\'ifica en de las instalaciones del Gimnasio. Ejemplo: Pesas, Yoga, Natación, etc.\\

\textbf{Nivel de Actividad:} Grado de avance o complejidad dentro de alguna actividad.

\section{Objetivos del sistema}
Se pretende desarrollar un sistema inform\'atico que gestione aquellas funciones necesarias para laa automatizaci\'on de las actividades del Gimnasio Atl\'etico San Pancho. Estas actividades cubren ell ciclo de vida de cada operaci\'on desde que el cliente se da de alta en el sistema hasta que se da de baja, pasando por todos los servicios que el gimnasio provee.
\clearedoublepage
